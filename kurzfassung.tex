
\chapter{Kurzfassung}
\begin{german}
Mobilitätssimulationen sind ein wichtiges Instrument um den Betrieb von mobilen Kommunikationssystemen zu garantieren. Der Zweck von Mobilitätssimulationen ist herauszufinden wie mobile Benutzer das Netzwerk beeinflussen. Üblicherweise werden diese Mobilitätsmodelle von stochastischen Prozessen oder Umfragen abgleitet. Ein weiterer Ansatz ist es, diese Modelle aus echten Informationen der Mobilfunkbenutzer abzuleiten. Dies erlaubt dem Netzwerkbetreiber sein Netzwerk mit Bewegungsmodellen seiner Mobilfunknutzer zu simulieren.

In dieser Arbeit wir ein System vorgestellt zur Erzeugung von Trajektorien für Mobilfunkbenutzer. Die hierdurch erzeugten Trajektorien können anschließend in Mobilsimulationen verwendet werden. Das vorgestellte System untersucht Handover Events. Da GSM Handover nur durchführt wenn das Mobiltelefon im "`connected"' Modus ist, können Trajektorien nur für Mobilfunkbenutzer erzeugt werden welche gerade Telefonieren. Es wird ein Ansatz präsentiert welcher mit Hilfe von Prädiktionsrechnungen die Position des Handover ermittelt. Das System verwendet Bevölkerungsdichtekarten um die Start- und Endposition des Mobilfunkbenutzers zu schätzen. Die erzeugten Trajektorien bestehen aus der gefahrenen Route und Zeitstempeln welche Rückschlüsse auf die Geschwindigkeit geben.
\end{german}