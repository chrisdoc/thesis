\chapter{Summary}
In this thesis a trajectory generation pipeline for the automated generation of mobile subscriber trajectories has been presented. Mobility simulations are an important aspect in mobile network simulations, since they allow to investigate the effects mobile subscriber have on the mobile network. In the recent years real world behavior models have gained more and more importance in the field of mobility analysis, because they allow simulations in a real environment.

The developed system is using call detail records to estimate driving trajectories for mobile subscribers. The call detail records are used together with coverage predictions and a representation of the road network to estimate the trajectory.
Trajectories are generated for subscribers during an active call. Therefore, the location of the call establishment marks the begin of the trajectory. Because GSM does not expose an accurate subscriber location an approximation takes place which takes the population density of the area into account.
To derive the average subscribers velocity between two consecutive handover, the presented system calculates a most likely handover position. In order to make the generated trajectories accessible to third-party applications such as mobility simulation framework a simple output format has been chosen. The output format persists the trajectory as a list of latitude and longitude coordinates as well as an average velocity that defines the subscribers movement.
\section{Résumé}
This thesis gives an introduction into GSM in order to provide a better understanding of the working principle of the trajectory generation. Besides that, several approaches of other researches in the field of trajectory generation for mobile subscribers are presented.
Based on these fundamentals, the trajectory generation pipeline is explained in detail. First, the input data required by the system will be described. This involves the subscriber information, the road network and the network coverage prediction. Two different types of network coverage prediction techniques are described and compared with each other. Afterwards, each step involved in the generation process is outlined, consisting of the call detection, route computation, route validation, handover estimation and timing annotation.   
\section{Discussion}
At this time, the developed system can be used to estimate driving trajectories for mobile subscriber based on the subscriber information. The estimated routes correspond to the route driven by the subscriber. As shown in chapter~\ref{cha:results} trajectories can be estimated for subscribers in urban and semi-rural areas. As seen in the example a major drawback of this approach is the unpredictability of the handover points. Both coverage perdiction methods have shown that there exist a deviation between the estimated handover point and the observed one. Therefore enhancement would be to use coverage predictions provided by the network operators. This coverage predictions shall better represent the coverage area. Additionally different coverage predictions can be used for different areas. Therefore, the accuracy of the handover position can be increased. This, however, only affects the timing information, the route finding process is not affected.
To overcome this deviation we presented an approach which adapts the handover position. This approach eliminates the velocity overrun and produces reliable velocity values that correspond to the actual velocity of the subscriber.

The greatest limitation of that system is that handover events in GSM are only made when the subscriber is in connected mode. With the increase usage of data services on smartphones, a possible extension would be to investigate routing area updates and handover updates in the GPRS, UMTS and LTE network. Right now these events can not be used because they terminate in the radio access network and are not forwared to the core network where the monitoring units are placed.
